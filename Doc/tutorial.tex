% -*-latex-*-
\chapter{Tutorial\label{ch.tutorial}}

The package can be used either to generate simulated data sets for
given parameter values (see documentation of \io{mmgen} below) or to
analyze data in the form of a ``mismatch distribution''.  This
tutorial concerns the latter problem.

\section{Generating a mismatch distribution from sequence data}

If you are beginning with sequence (or restriction site) data, the
first step is to convert these into a mismatch distribution.  If your
data are in (noninterleaved) PHYLIP format, the program \io{makemm} will
do this for you.  This program reads sequence data such as that shown
below: 
\begin{verbatim}
file: dummy.seq
  10  20
subject1  TGAAT TTGTC
          CCCGC CCTCT
subject2  TGAAT TTGTC
          CCTAC ACCCT
subject3  TAGAT CTGTC
          TCTGA CCTCT
subject4  TGAAT TTGTC
          TCTGA CCTCT
subject5  TGAAT TTGCC
          ACCGC ACCCT
subject6  AAGCT TCGTC
          CTCAC GCTCT
subject7  TGAAT TTGCC
          CCCGC ACCTT
subject8  TGACT TTGCC
          CGCGC ACCCT
subject9  TAGAT TTGTC
          CCCGC GCTCC
subject10 TGAAT TTGCC
          CCCTC ACCTA
\end{verbatim}
The first line of this data tells the program that there are 10
subjects, each with data at 20 sites.  The first 10 characters of the
next line give the name of the 1st subject.  After the name, the
program expects to find sequence data.  In reading sequence data, it
ignores spaces, tabs, and linefeeds so that sequences can be spread
across several lines or broken by vertical columns of whitespace.

To run the program on the data in file \io{dummy.seq}, type
\begin{verbatim}
makemm dummy.seq
\end{verbatim}
or
\begin{verbatim}
makemm < dummy.seq
\end{verbatim}

The program writes the following result on the standard output:
\begin{verbatim}
%                                   makemm
%                       (make a mismatch distribution)
%                             by Alan R. Rogers
%                                Version 4-1
%                                12 Nov 1998
%                         Type `makemm -- ' for help

% Cmd line: makemm dummy.seq
% pop  0 : dummy.seq
% Results will include the reference sequence (line 1).

% Population 0
% 10 subjects 20 sites
% mean pairwise diff = 6.622222.
nsequences = 10 ; 
nsites =  20 ;
mismatch = 0 0 2 4 7 5 5 5 5 4 2 4 2 ;
segregating = 16 ;
\end{verbatim}
Here, ``mean pairwise diff'' is the average number of nucleotide (or
restriction) site differences between pairs of sequences,
``nsequences'' is the number of sequences, ``nsites'' is the number of
sites, and ``segregating'' is the number of these sites that were
segregating (i.e.\ the number at which some variation occurred.  The
mismatch distribution is a vector of integers.  The first entry is the
number of pairs of individuals that differ by 0 sites, the second is
the number that differ by 1 site, and so forth.

By default, these data are printed in the format that is needed by the
next program to be considered, \io{mmest} (see below).  Several
command-line switches are available for changing the format of the
output.  To get a list of these, type \io{mmest --}.

\section{Creating an input file for \io{mmest}}

The output of \io{makemm} is already in the right format.  To save a
copy of the output for further analysis, type \io{mmest > dummy.mm}.
This creates a new file called \io{dummy.mm}.  If you want to
construct your own input file, you need to know that \io{mmest}
ignores everything between the character ``\%'' and the end of the
line.  This allows you to put comments into your input files.

There is one minor irritant: If you are analyzing sequence data, the
effective number of sites is the number of sites in the data.  But if
you are analyzing restriction site data, the effective number of sites
is \emph{twice} the number of sites in the data.  (See Ewens
\cite{Ewens:RoleOfModels} and Rogers and Harpending
\cite{Rogers:MBE-9-552} for a discussion of this confusing fact.)
\io{makemm} is incapable of telling the difference.  Consequently,
with restriction site data you must edit the output of \io{makemm} and
double the value of \io{nsites}.

All assignment statements are of form
\begin{verbatim}
  <label> = <values> ;
\end{verbatim}
where \verb@<label>@ is a name and \verb@<values>@ a blank-delimited
list of one or more values.  These values are usually numbers, but in
some contexts (described later) they will be character strings.

\section{Estimating parameters with \io{mmest}}

The program \io{mmest} is used to estimate parameters from data.  To
invoke it with the data in \io{dummy.mm}, type \io{mmest dummy.mm}.
This produces the following output:
\begin{verbatim}
%                                   MMEST
%                  (Estimation from Mismatch Distribution)
%                             by Alan R. Rogers
%                                Version 4-1
%                                12 Nov 1998
%                         Type `mmest -- ' for help

% Cmd line: mmest dummy.mm
%Using histogram of length 13. Input histogram had length 13.
InputFile = dummy.mm;
Sampsize = 10;
NSites = 20;
Histogram = 0 0 2 4 7 5 5 5 5 4 2 4 2;

%                       mean           var E[(x-mean)^3]
   Cumulants =       6.62222       7.74617       5.65513 ;

%       theta0    theta1       tau       MAE     Rghns       Seg
Est=      1.06       Inf     5.562    0.1261   0.01679        16;

%mmci ranges are specified by: start_val, end_val, increment
%Edit the following to control mmci:
RangeLog10Theta0 = 0 3 0.50 ;
RangeGrowth = 0 3 1 ;
RangeTau = 2 12 2 ;
\end{verbatim}
After the header, several lines of output echo the data that went into
the analysis.  The next line provides the first three ``cumulants'' of
the distribution: the mean, the variance, and the average cube of $x -
\bar x$.  Then comes a line of estimates.  The estimates provided can
be controlled by editing the top section of the header file
\io{mismatch.h} and recompiling.  As distributed, the estimates
include:
\begin{description}
\item[$\THETAa$]the 2-parameter method of moments estimator described by
          Rogers (1995a).
\item[$\THETAb$]estimated (very roughly) as $(1/F[0]) - 1$, where
      $F[0]$ is the first entry of the empirical mismatch distribution
      \cite{Rogers:MBE-9-552}. DO NOT PLACE ANY FAITH IN THIS
          ESTIMATOR; it provides only a very rough indication of the 
          magnitude of $\THETAb$.  
\item[$\TAU$]the 2-parameter method of moments estimator described by
          Rogers \cite{Rogers:E-49-608}.
\item[$\MAE$] Mean Absolute Error.  The mean absolute difference
      between the fitted curve and the empirical distribution.  The
          empirical distribution is $h/\Sum{h}$ where $h$ is the histogram
          in the input data.  The fitted curve is calculated from
          equation~4 of Rogers and Harpending \cite{Rogers:MBE-9-552}. 
\item[Rghns]Harpending's (1994) $r$ statistic, a measure of roughness.
\item[Seg]The number of segregating sites, copied from the input.
\end{description}
Following the estimates come several lines that will become important
later, in the discussion of \io{mmci}.

To save a copy of the estimation output (under UNIX or DOS), type
\begin{verbatim}
         mmest dummy.mm > csw.est
\end{verbatim}
This places the estimation output into a file called \io{dummy.est}.

\section{Constructing a confidence interval}

To construct a confidence interval, you will need the file
\io{dummy.est} that you produced using \io{mmest}.  In addition, you
will need to describe your population's history by creating a file
called \io{pophist.ini}.  This latter file should look something like
\begin{verbatim}
   %theta      M     tau        K
   1000        0       2        3
      1       10     Inf        1
\end{verbatim}
Here, each row describes a different ``epoch'' of population history,
with the most recent epoch at the top, the earliest at the bottom.
Your file can have as many epochs as you want.  The four colums are:
\begin{definition}
\item[$\THETA$]$2Nu$, where $N$ is the female population size
      (including all subpopulations combined) and $u$ the aggregate
      mutation rate.

\item[$M$]the number of migrants per generation between each pair of  
       subpopulations.

\item[$K$]the number of subpopulations
 
\item[$\TAU$]$2ut$, where $t$ is the length of the epoch in
      generations.  $\TAU$ measures the epoch's length in
      ``mutational'' time units, which equal $1/(2*u)$ generations.
\end{definition}
The initial value of \io{tau} (at the bottom) should be ``Inf,'' to
indicate that the initial epoch is infinite.  See
chapter~\ref{ch.simulate} for further information.

The simplest way to construct a confidence interval is to type
\begin{verbatim}
  mmci dummy.est > dummy.ci
\end{verbatim}
This invokes the program \io{mmci}, which will read \io{csw.est} and
\io{pophist.ini}, construct a confidence interval, and place the
result into file \io{csw.ci}.  This uses the default settings, and
will do 1000 iterations at each set of parameter values.  This will
take awhile to execute.  In initial experiments, you may want to type
something like
\begin{verbatim}
  mmci -i100 dummy.est > dummy.ci
\end{verbatim}
which will generate a low-quality confidence interval more quickly.
The \io{-i} flag tells mmci how many iterations to do at each set of
parameter values.

After looking at the resulting confidence interval, you will probably
wish that mmci had looked a different range of parameter values.  To
specify your own range, you have to edit the file \io{dummy.est}, adding
or modifying lines such as the following:
\begin{verbatim}
 RangeLog10Theta0 = -1.6666667 1.6666667  0.33333333 ;
 RangeGrowth = 0 5  1 ;
 RangeTau    = 1 7  1 ;
\end{verbatim}
The three lines specify ranges for the following variables:
\begin{description}
\item[\io{Log10 theta0}] This is the log10 of the parameter theta0 that is
                 discussed above.

\item[\io{Growth}] Equals $\CLog(\THETAb/\THETAa)$, the log of the
      ratio of the new population size to the old.

\item[\io{Tau}] the time since the change in population size, on a
      mutational time scale (see above) 
\end{description}
The three numbers in each line specify (1)~the initial value, (2)~the
final value, and (3)~the increment.  For example, the first line says
that \io{Log10Theta0} will range from --1.667 to 1.667, in increments
of 0.33.  The last line says that $\TAU$ will take values
$1,2,\ldots,7$.

\io{Mmci} does many thousands of simulations and takes a long time to
finish.  Before embarking on a full-scale run, there are some quick
checks you can make to ensure that you have specified things
correctly.   First type
\begin{verbatim}
  mmci -i0 dummy.est 
\end{verbatim}
This tells mmci to simulate 0 data sets at each set of parameter
values, i.e.\ to do nothing at all.  This is useful for making sure
that you have the parameter ranges right.  Next, try
\begin{verbatim}
  mmci -i10 dummy.est > dummy.ci
\end{verbatim}
This simulates 10 data sets at each set of parameter values and
produces a low-quality estimate of the confidence region.  Plot the
result using \io{ci2txt}, as explained below.  The plot may suggest
that you need to examine a different part of the parameter space.  If
so, then edit the parameter ranges in your \io{.est} file and run
\io{mmci} again.  When you are satisfied with the parameter ranges, type
\begin{verbatim}
  mmci > dummy.ci
\end{verbatim}
This will simulate 1000 data sets for each set of parameter values,
and it may take a long time.  If you are working on a slow machine,
you may have to settle for a somewhat smaller analysis.  Try
\io{-i250}. 

The output of \io{mmci} is a list of hypothesis tests.  This output is
self-explanatory, but poring over it is a waste of time.  To find out
what your confidence region looks like, you should plot it using one
of the programs described in the next section.

\section{Plotting a Confidence Interval}

There are two ways to make a graphical display of the output.  Program
\io{ci2txt} produces a low-quality character plot; program \io{ci2ptx}
produces a high-quality plot using \LaTeX\ and \PiCTeX, two programs
that may or may not be available on your system
\cite{Lamport:LDP-94,Wichura:PM-87}.  \LaTeX\ and \PiCTeX\ are both
available for free over the net, but learning to use them takes time.
This tutorial will concentrate on \io{ci2txt}.  For more information
on \io{ci2ptx}, see chapter~\ref{ch.ci2ptx}.

To graph the confidence region, type 
\begin{verbatim}
ci2txt dummy.ci
\end{verbatim}
This generates a series of graphs, one for each level of the growth
parameter.  The first of these looks like this:
\begin{alltt}
\begin{minipage}{\textwidth}
Definitions:
 O : Reject at 1 percent significance level
 . : Within 99 percent confidence region
 * : Within 95 percent confidence region


Panel 0: N1/N0 = 10^0:

l          +-----------+
o    3.00  |      O    |
g    2.50  |      O    |
     2.00  |      O    |
t    1.50  |      O    |
h    1.00  |      *    |
e    0.50  |      *    |
t    0.00  |      O    |
a          +-----------+
0                       
              undefined 
\end{minipage}
\end{alltt}
This panel summarizes simulations for the case in which no growth
occurred, so that $\THETAa = \THETAb$.  The vertical axis shows values
of $\CLog \THETAa$.  The horizontal axis is usually for $\TAU$, but it
is undefined here because this panel is for the case in which no
growth has occurred.  Within the plot, each symbol indicates the
result of a hypothesis test.  Open circles ``O'' indicate hypotheses
that have been rejected at the 0.01 significance level, dots ``.''
indicate hypotheses that have been rejected at the 0.05 significance
level, and asterisks ``*'' indicate hypotheses that have not been
rejected.   The plot shown here indicates that we cannot reject the
hypotheses that $\CLog \THETA = 0.5$ or that $\CLog \THETA = 1$.  All
other hypotheses involving a constant population size are rejected.

The second panel in the output of \io{ci2txt} looks like
\begin{alltt}
\begin{minipage}{\textwidth}
Panel 1: N1/N0 = 10^1:

l          +-----------+
o    3.00  |O O O O O O|
g    2.50  |O O O O O O|
     2.00  |O O O O O O|
t    1.50  |O O O O O O|
h    1.00  |* * * . O O|
e    0.50  |* * * * * O|
t    0.00  |O * * * * *|
a          +-----------+
0                   1 1 
            2 4 6 8 0 2 
                        
                    tau 
\end{minipage}
\end{alltt}
This panel summarizes hypothesis tests for cases in which $\THETAb$ is
10 times as large as $\THETAa$ (and therefore $\Nb$ is 10 times as
large as $\Na$).  Here, the horizontal axis indicates values of
$\TAU$.  As before, asterisks indicate the portion of the parameter
space that is contained within the confidence region.
