% -*-latex-*-
\chapter{Program: \io{mmci}\label{ch.mmci}}

\section{Purpose} Estimate a confidence region.

\section{Usage\label{sec.mmci.usage}} \io{mmci [options] inputfile}\\
where options may include:
\begin{description}
\item[\io{-g<x>}] Set gamma shape parameter to \verb@<x>@ Default=1.000000
\item[\io{-i<x>}] Set iterations to \verb@<x>@. Default=1000
\item[\io{-l<x>}] Set length of mismatch distribution to
\verb@<x>@. Default=as in input  
\item[\io{-mi}]   mutation model = infinite sites
\item[\io{-mf}]   mutation model = finite sites w/ equal rates
\item[\io{-mg}]   mutation model = finite sites w/ gamma rates
\item[\io{-ms}]   mutation model = stepwise
\item[\io{-r}]    Reset gamma-model mutation rates each time? Default=NO
\item[\io{-v}]    Toggle verbose mode Default=YES
\end{description}
See section~\ref{sec.mutation} for a description of mutational models.

\section{Description}

\io{mmci} considers a large number of population histories.  At each
history, it uses computer simulations (see ch.~\ref{ch.simulate}) to
perform a significance test.  The confidence region consists of the
union of histories that cannot be rejected.

\subsection{Testing a population history\label{sec.mmci.test}}

The following steps are performed for each population history:
\begin{enumerate}
\item Calculate the theoretical mismatch distribution from the
history.

\item Calculate the Mean Absolute Error $(\MAE)$ between the observed
mismatch the theoretical mismatch distributions.  Call this the
``observed $\MAE$.''

\item Simulate a large number of data sets using this same history, as
described in chapter~\ref{ch.simulate}.

\item For each simulated data set, calculate the $\MAE$ between the
simulated mismatch distribution and the theoretical mismatch
distribution.  Call these ``simulated $\MAE$s.''

\item Reject the history at the 0.05 significance level if the
observed $\MAE$ is greater than 95 percent of the simulated $\MAE$s. 
\end{enumerate}
At present, the program ignores population structure in calculating
the theoretical distribution.  This reflects my original purpose in
adding population structure to the program---I wanted to ask how much
error is introduced when the real population is structured, but the
analysis assumes global random mating \cite{Rogers:PPG-97-55}.  The
confidence regions are still valid but could probably be made smaller
(at least in structured populations) by incorporating population
structure into the analysis.  It is not easy to incorporate population
structure because we have no simple formula, analogous to the one in
Rogers and Harpending \cite{Rogers:MBE-9-552}, for the theoretical
mismatch distribution of a subdivided population.

\subsection{Models of mutation\label{sec.mmci.mutation}}

Several models of mutation are available: infinite sites, finite sites
with equal rates, finite sites with gamma-distributed rates, and
stepwise.  These are described in chapter~\ref{ch.simulate}.
The choice between them is made by command-line switches, as described
in section~\ref{sec.mmci.usage}.

\section{Input}

There are three sorts of input to worry about.  

\subsection{Command line arguments}

The operation of \io{mmci} can be controlled through a variety of
command line arguments, as summarized in section~\ref{sec.mmci.usage}
above.

\subsection{The primary input file}

The primary input file, which must be specified on the command line,
has a format just like the output file produced by \io{mmest}.  The
distribution contains a file that was generated
by applying \io{mmest} to the Cann-Stoneking-Wilson data in file
\io{csw.mm}.  This file is called \io{csw.est} and looks like this:
\begin{verbatim}
%                                   MMEST
%                  (Estimation from Mismatch Distribution)
%                             by Alan R. Rogers
%                                Version 4-2
%                                16 Nov 1998
%                         Type `mmest -- ' for help

% Cmd line: mmest csw.mm
% Using histogram of length 31. Input histogram had length 31.
InputFile = csw.mm;
NSequences = 147;
NSites = 3000;
Mismatch = 28 49 126 245 402 640 874 1178 1245 1119 1084 874 748 619
  451 336 217 150 87 42 42 42 21 21 21 21 14 14 7 7 7;

%                       mean           var E[(x-mean)^3]
   Cumulants =        9.4706       16.2234       60.3741 ;

%       theta0    theta1       tau       MAE     Rghns       Seg
Est=     2.599     382.2     6.872  0.002538  0.003594       195;

% mmci ranges are specified by: start-val, end-val, increment
% Edit the following to control mmci:
RangeLog10Theta0 = 0 3 0.50 ;
RangeGrowth = 0 3 1 ;
RangeTau = 2 12 2 ;
\end{verbatim}
When \io{mmci} reads this file, it treats the \%{} as a comment
delimiter, ignoring everything from the \%{} to the end of the line.
All the other lines are in the form of an assignment statement:
\begin{verbatim}
  <label> = <values> ;
\end{verbatim}
where \verb@<label>@ is a name and \verb@<values>@ a blank-delimited
list of one or more values.  The meaning of all this is explained in
chapter ~\ref{ch.mmest}.

The last section of the input file above is used to control the
population history parameters that are considered.  Although the
history may have many parameters, only 3 of these are varied
systematically by \io{mmci}.  To vary the others, you have to run the
program more than once, changing the \io{pophist.ini} file before each
run.  The variable parameters are:
follows:
\begin{description}
\item[$\THETAb$] the value of $\THETA$ in the most recent epoch\\
\item[$\THETAa$] the value of $\THETA$ in the preceding epoch\\
\item[$\TAU$] the length of the most recent epoch in mutational time
units
\end{description}

To govern the behavior of the three variable parameters, you must
specify three range variables: \io{RangeLog10Theta0},
\io{RangeGrowth}, and \io{RangeTau}.  Here, ``Growth'' is the common
log of $\THETAb/\THETAa$.  Each range variable is specified by a
statement of form
\begin{center}
\io{Range = start-value end-value increment ;}
\end{center}
For example,
\begin{verbatim}
RangeLog10Theta0 = 0 2 0.5 ;
\end{verbatim}
tells \io{mmci} that $\CLog \THETA_a$ should range from 0 through
2 in increments of 0.5.  

\subsection{The population history file}

The \io{pophist.ini} file is described in detail in
section~\ref{sec.pophist}.  The discussion here will cover only the
peculiarities in the way \io{mmci} treats this file.

Suppose that my \io{pophist.ini} file looked like this:
\begin{verbatim}
%theta      M       tau        K
1000        0       2          3
0.1         0       0.3        1
100         10      Inf        1
\end{verbatim}
\io{mmci} would pay no attention to the values
$\THETA[2] = 0.1$, $\THETA[3] = 1000$, and $\TAU[3]=2$.  Instead,
these values would be overridden by the ranges specified in the input
file.   The rest of \io{pophist.ini}, however, would be used in the
simulations. 

\section{Output}

To produce some sample output, I used the values of the preceding
examples to set the range parameters in the file
\io{csw.est} and the history parameters in the file \io{pophist.ini}
and then executed \io{mmci} by typing
\begin{verbatim}
  mmci csw.est > csw.ci
\end{verbatim}
This places the primary output of \io{mmci} into the file \io{csw.ci}.
However, \io{mmci} continues to write messages to your screen so that
you can tell how it is progressing.  This is important because one can
easily launch \io{mmci} into a run that would take months to complete.
You need a way of estimating the time that will required.  After
\io{mmci} had been running for a few minutes, my screen looked like
this: 
\begin{verbatim}
[rogers@willow]\$ mmci csw.est > csw.ci

Reading history from file pophist.ini
% tau values: 0.300000 2.000000 5.000000 6.000000
% theta values: 0.1 1 10 100 1000 10000 100000
14548 bytes allocated to stack 1
Doing 14000 iterations in all
progress: 36.20%
\end{verbatim}
The bottom line is updated after every simulation and makes it easy to
estimate the time that the whole job will take.  On my computer, it
took 17~s for \io{mmci} to complete 1 percent of the job, which
implies that the whole job will take about 28 minutes.  This is pretty
fast, and reflects the speed of the infinite sites model of mutation.
The same experiment with the Gamma mutational model 
(\io{mmci~-mg~...}) would take about two hours.  If your data include
lots of subjects and lots of sites and your range specification calls
for simulations at lots of parameter values, you may find that your
runs require weeks of computer time.

\io{mmci}'s primary output file starts by echoing the data that went
into it:
\begin{verbatim}
%                                    MMCI
%              (Confidence Interval for Mismatch Distributions)
%                             by Alan R. Rogers
%                                Version 4-2
%                                16 Nov 1998
%                          Type `mmci -- ' for help

%     theta         mn        tau          K
% 1000.0000     0.0000    70.0000          1
%   10.0000     0.0000        Inf          1

% Using histogram of length 31. Input histogram had length 31.
%InputFile = csw.mm;
%Iterations = 1000;
% total sampsize=147 subdivisions=1
% subdivision sizes: 147
%Sampsize = 147;
% Mutation model = infinite sites
% Tests use: MAE
% Observed values: MAE=0.002538;
RangeLog10Theta0 = 0.000000 3.000000 0.500000 ;
RangeGrowth = 0.000000 3.000000 1.000000 ;
RangeTau = 2.000000 12.000000 2.000000 ;
\end{verbatim}
I hope that this material is self explanatory.

After this introductory section comes the confidence region.  This is
presented as a series of tests, one on each line:
\begin{verbatim}
%%%%%%%%%%%%%%%Confidence Interval%%%%%%%%%%%%%%%

% Begin figure 1
    Growth =       0.00000 ;
%                      tau log10[theta0]         p-val
      Test =     5.0000000     0.0000000     0.0000000 ;
      Test =     5.0000000     2.0000000     0.0000000 ;

% Begin figure 2
    Growth =       1.00000 ;
%                      tau log10[theta0]         p-val
      Test =     5.0000000     0.0000000     0.0000000 ;
      Test =     5.0000000     2.0000000     0.0000000 ;
      Test =     6.0000000     0.0000000     0.0030000 ;
      Test =     6.0000000     2.0000000     0.0000000 ;
\end{verbatim}
and so on.  Each section begins with a \io{Growth} value, which is
equal to $\CLog (\THETAb/\THETAa)$.  Within each section is a series
of lines labeled \io{Test}, each of which shows the outcome of a
particular test.  The first test in this output considers the case in
which $\TAU=5$, $\CLog \THETAa = 0$, and $\CLog (\THETAb/\THETAa)=0$.
For this test, the estimates $p$-value is 0, which means that all of
the simulated $\MAE$s were smaller than the observed $\MAE$ (see
section~\ref{sec.mmci.test}).  The 95 percent confidence region
consists of the union of the population histories for which the
$p$-value is at least 0.05.  In the output above, none of the lines
shown would be inside the confidence region.

To make this output useful, we need some means of turning it into a
graphical display.  For this purpose, see the documentation of
\io{ci2txt} and \io{ci2ptx}. 

