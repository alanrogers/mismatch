% -*-latex-*-
\chapter{Program: \io{ci2ptx}\label{ch.ci2ptx}}

\section{Purpose} Plot a confidence region using \PiCTeX\ graphics

\section{Usage\label{sec.ci2ptx.usage}} \io{ci2ptx [options] [inputfile]}\\
where options may include:
\begin{description}
\item[\io{-h$<$number$>$}] Set plot height to \verb@<number>@ inches.
\item[\io{-w$<$number$>$}] Set plot width to \verb@<number>@ inches.
\end{description}
By default, the program reads standard input.

\section{Description}

\io{ci2ptx} reads an input file describing a confidence region, which
must be in the format of the output produced by \io{mmci}, and writes
a series of \PiCTeX{} commands to the standard output.  These commands
generate a graphical display of the confidence region when processed
by the text-formatting program \LaTeX\ \cite{Lamport:LDP-94} together with the
\PiCTeX{} macros \cite{Wichura:PM-87}.  These are both available from
the address listed in the bibliography.

To produce a graphical display, proceed as follows:
\begin{enumerate}
\item Execute the command 
\begin{verbatim}
  ci2ptx csw.ci > figxxx.tex
\end{verbatim}
to place \PiCTeX{} code into the file \io{figxxx.tex}.  

\item Copy the file mkfig.tex from the directory that contains the
    mismatch source code into your current directory.  \io{mkfig.tex} is
    a very brief file that tells \LaTeX{} to read in the file
    \io{figxxx.tex}, which you just created.

\item Make sure that your current directory contains a file called
      \io{mkfig.tex} containing the following:
\begin{verbatim}
% -*-latex-*-
\documentclass[]{article}
\usepackage{pictex}
\begin{document}
% -*-latex-*-
%InputFile = csw.ci2
%RangeLog10Theta0: -1.500000 1.500000 0.500000 
%RangeGrowth: 0.000000 3.000000 1.000000 
%RangeTau: 2.000000 12.000000 2.000000 
%% Confidence Interval: PicTeX output %%
%% Definitions:
\def\accept{{$\bullet$}}
\def\99pct{{$\cdot$}}
\def\reject{{\footnotesize$\circ$}}
%% Use the following to hide the 1% region
\def\99pct{\reject}
%%Plots are 1.000000 inch wide and 0.500000 inches high
\begin{figure}
\begin{center}
\mbox{%
\beginpicture
\headingtoplotskip=0.5\baselineskip
\valuestolabelleading=0.4\baselineskip
%%%%%%%%%%%% Plot figure in row 0 col 0 %%%%%%%%%%%%
\setcoordinatesystem units <0.100000in,0.166667in> point at 22.000000 1.500000
\setplotarea x from 2.000000 to 12.000000, y from -1.500000 to 1.500000
\axis bottom shiftedto y=-1.800000 /
\axis left shiftedto x=1.500000 label {$\theta_0$}
   ticks withvalues 0.1 1 10 /
     at -1.000000 0.000000 1.000000 / /
\plotheading{\small No growth}
%                        tau log10[theta0]
\put   {\reject} at        7     -1.500000   %p-val=0
\put   {\reject} at        7     -1.000000   %p-val=0
\put   {\reject} at        7     -0.500000   %p-val=0.004
\put    {\99pct} at        7      0.000000   %p-val=0.018
\put   {\reject} at        7      0.500000   %p-val=0.008
\put   {\reject} at        7      1.000000   %p-val=0.002
\put   {\reject} at        7      1.500000   %p-val=0
%%%%%%%%%%%% Plot figure in row 0 col 1 %%%%%%%%%%%%
\setcoordinatesystem units <0.100000in,0.166667in> point at 2.000000 1.500000
\setplotarea x from 2.000000 to 12.000000, y from -1.500000 to 1.500000
\axis bottom shiftedto y=-1.800000 label {$\tau$}
   ticks withvalues 2.5 5 7.5 10 /
     at 2.500000 5.000000 7.500000 10.000000 / /
\axis left shiftedto x=1.500000 label {$\theta_0$}
   ticks withvalues 0.1 1 10 /
     at -1.000000 0.000000 1.000000 / /
\plotheading{\small$10^{1}$-fold growth}
%                        tau log10[theta0]
\put   {\reject} at        2     -1.500000   %p-val=0
\put    {\99pct} at        2     -1.000000   %p-val=0.016
\put   {\reject} at        2     -0.500000   %p-val=0.004
\put   {\reject} at        2      0.000000   %p-val=0
\put   {\reject} at        2      0.500000   %p-val=0.002
\put   {\reject} at        2      1.000000   %p-val=0
\put   {\reject} at        2      1.500000   %p-val=0
\put   {\reject} at        4     -1.500000   %p-val=0
\put    {\99pct} at        4     -1.000000   %p-val=0.022
\put    {\99pct} at        4     -0.500000   %p-val=0.01
\put   {\reject} at        4      0.000000   %p-val=0
\put   {\reject} at        4      0.500000   %p-val=0
\put   {\reject} at        4      1.000000   %p-val=0
\put   {\reject} at        4      1.500000   %p-val=0
\put   {\reject} at        6     -1.500000   %p-val=0
\put   {\accept} at        6     -1.000000   %p-val=0.082
\put    {\99pct} at        6     -0.500000   %p-val=0.022
\put   {\reject} at        6      0.000000   %p-val=0.002
\put   {\reject} at        6      0.500000   %p-val=0
\put   {\reject} at        6      1.000000   %p-val=0
\put   {\reject} at        6      1.500000   %p-val=0
\put   {\reject} at        8     -1.500000   %p-val=0
\put   {\accept} at        8     -1.000000   %p-val=0.102
\put    {\99pct} at        8     -0.500000   %p-val=0.038
\put    {\99pct} at        8      0.000000   %p-val=0.01
\put   {\reject} at        8      0.500000   %p-val=0
\put   {\reject} at        8      1.000000   %p-val=0
\put   {\reject} at        8      1.500000   %p-val=0
\put   {\reject} at       10     -1.500000   %p-val=0
\put   {\accept} at       10     -1.000000   %p-val=0.096
\put   {\accept} at       10     -0.500000   %p-val=0.098
\put   {\reject} at       10      0.000000   %p-val=0
\put   {\reject} at       10      0.500000   %p-val=0.002
\put   {\reject} at       10      1.000000   %p-val=0
\put   {\reject} at       10      1.500000   %p-val=0
\put   {\reject} at       12     -1.500000   %p-val=0
\put   {\accept} at       12     -1.000000   %p-val=0.058
\put   {\accept} at       12     -0.500000   %p-val=0.244
\put   {\reject} at       12      0.000000   %p-val=0.004
\put   {\reject} at       12      0.500000   %p-val=0
\put   {\reject} at       12      1.000000   %p-val=0
\put   {\reject} at       12      1.500000   %p-val=0
%%%%%%%%%%%% Plot figure in row 1 col 0 %%%%%%%%%%%%
\setcoordinatesystem units <0.100000in,0.166667in> point at 22.000000 9.000000
\setplotarea x from 2.000000 to 12.000000, y from -1.500000 to 1.500000
\axis bottom shiftedto y=-1.800000 label {$\tau$}
   ticks withvalues 2.5 5 7.5 10 /
     at 2.500000 5.000000 7.500000 10.000000 / /
\axis left shiftedto x=1.500000 label {$\theta_0$}
   ticks withvalues 0.1 1 10 /
     at -1.000000 0.000000 1.000000 / /
\plotheading{\small$10^{2}$-fold growth}
%                        tau log10[theta0]
\put   {\reject} at        2     -1.500000   %p-val=0.006
\put   {\reject} at        2     -1.000000   %p-val=0.002
\put   {\reject} at        2     -0.500000   %p-val=0
\put   {\reject} at        2      0.000000   %p-val=0
\put   {\reject} at        2      0.500000   %p-val=0
\put   {\reject} at        2      1.000000   %p-val=0
\put   {\reject} at        2      1.500000   %p-val=0
\put   {\reject} at        4     -1.500000   %p-val=0.008
\put   {\reject} at        4     -1.000000   %p-val=0.004
\put   {\reject} at        4     -0.500000   %p-val=0
\put   {\reject} at        4      0.000000   %p-val=0
\put   {\reject} at        4      0.500000   %p-val=0
\put   {\reject} at        4      1.000000   %p-val=0
\put   {\reject} at        4      1.500000   %p-val=0
\put    {\99pct} at        6     -1.500000   %p-val=0.014
\put   {\reject} at        6     -1.000000   %p-val=0.002
\put   {\reject} at        6     -0.500000   %p-val=0
\put   {\reject} at        6      0.000000   %p-val=0
\endpicture}
\end{center}
\caption{Confidence Region}
\end{figure}
       %% Edit this line to match your filename
\end{document}
\end{verbatim}
If you change the name of \io{figxxx.tex}, you must edit
\io{mkfig.tex} to reflect this change.

\item Type \io{latex mkfig} to run \LaTeX.  This produces a file called
    \io{mkfig.dvi}.

\item To look at the resulting plot on your screen, try typing
      \io{xdvi~mkfig} if you are working under X-windows.  If this
      doesn't work, or if you are working in some other environment,
      ask your local computer guru how to preview the ``.dvi'' files
      that are produced by \TeX.

\item On my system, the next step is to type \io{dvips~mkfig}.  This
is really two steps rolled into one.  First, \io{dvips} translates the
.dvi format into the format that my printer understands.  Second, it
passes the translation to the printer.  This may require two steps on
your system.
\end{enumerate}

\subsection{Customizing the figure}

If you don't like the way the figure looks, there are two things you
can do.

\subsubsection{Customizing with command line arguments}
 
To make the individual plots 0.75 by 0.75 inches, use the following
command:
\begin{verbatim}
  ci2ptx -w0.75 -h0.75 csw.ci > figxxx.tex
\end{verbatim}

\subsubsection{Editing the \PiCTeX{} file}

You can also change the way the figure looks by editing the \PiCTeX{}
code in \io{figxxx.tex}.  For example, the lines that read
\begin{verbatim}
%% Use the following to hide the 1% region
%\def\99pct{\reject}
\end{verbatim}
provide a means of making the 99 percent confidence region show on the
graph.  All you have to do is to remove the ``\%'' from the second of
these lines.  The lines that read
\begin{verbatim}
%% Definitions:
\def\accept{{$\bullet$}}
\def\99pct{{$\cdot$}}
\def\reject{{\footnotesize$\circ$}}
\end{verbatim}
provide a means of changing the symbols that are used to represent
points inside and outside of the confidence region.  Change these
definitions to whatever suits you.


