% -*-latex-*-
\chapter{Introduction\label{ch.intro}}

\section{Package contents}

The package contains the following programs:
\begin{description}
\item[\io{mmest}] estimate parameters
\item[\io{mmci}] construct a confidence region
\item[\io{mmgen}] general purpose simulation program
\item[\io{ci2txt}] read a confidence region (the output of mmci)
and re-display it using character graphics
\item[\io{ci2ptx}] read a confidence region (the output of mmci)
and produce a graphical display in PicTeX format. 
\item[\io{normdist}] Normalize a distribution so that it sums to 1, or
      invert this process to obtain a distribution of counts.
\item[\io{makemm}] Reads sequence data in PHYLIP format; writes a mismatch
      distribution.  
\end{description}

\section{Overview}

A \emph{Mismatch distribution} (or distribution of pairwise
differences) is a histogram whose i'th entry is the number of pairs of
individuals in the data who differ by i sites.  \io{Mismatch} is a
package of computer programs designed to analyze such data.

The package estimates parameters and tests hypotheses that relate to
various models of population history.  Central among these is the
``model of sudden expansion,'' which assumes that an initial population
was at mutation-drift equilibrium with size $N_0$, and then grew (or
shrank) rapidly to size $N_1$.  We observe it $T$ generations later.  The
behavior of the mismatch distribution can be predicted from three
parameters:
\begin{definition}
\item[$\THETAa$]$2 u N_0$, where $u$ is the mutation rate per
      generation, summed over all nucleotide sites that contribute to
      the analysis, 
\item[$\THETAb$]$2 u N_1$
\item[$\TAU$]$2 u T$
\end{definition}
The programs in this package estimate these parameters and construct a
multivariate confidence region for them.  In simulations, the program
allows the user to specify a complex population history including an
arbitrary number of changes in population size, in the number of
subpopulations, and in the rate of migration between subpopulations.
The simulation programs can also accomodate a variety of assumptions
about the mutational process.  The statistical methods are described
in full by Rogers
\cite{Rogers:E-49-608,Rogers:MBE-13-895,Rogers:PPG-97-55}.   

\section{Getting information over the net}

Unpublished papers describing these methods can be obtained
electronically via anonymous ftp from anthro.utah.edu.  See the file
\io{ftp.doc}, which is included with the distribution.

\section{Installation}

See the file named \io{README}.

