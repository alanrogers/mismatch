% -*-latex-*-
\chapter{Simulations\label{ch.simulate}}

This chapter describes the algorithm used in simulations, a variant of
the geographically structured coalescent algorithm described by Hudson
\cite{Hudson:OSE-7-1}.  

\section{Describing a population history\label{sec.pophist}}

Before performing a simulation, you must first describe your
assumptions about the population's history.  Population histories may
be simple or complex and are described by creating a file called
\io{pophist.ini}, which must reside in the current directory at the
time a program is invoked.

I divide the past into an arbitrary number of epochs, within which
parameter values are constant.  Each epoch is described by four
parameters: $\THETA$, $M$, $\TAU$, and $K$, where
\begin{definition}
\item[$\THETA$]$2Nu$, where $N$ is the female population size
      (including all subpopulations combined) and $u$ the aggregate
      mutation rate (i.e. the sum of nucleotide mutation rates over
      the genomic region under study).  See Rogers \& Harpending
      \cite{Rogers:MBE-9-552} for further discussion of $u$. 

\item[$M$]the number of migrants per generation between each pair of  
       subpopulations (also equal to $Nm$ where $m$ is the migration
       rate) 

\item[$\TAU$]$2ut$, where $t$ is the length of the epoch in
      generations.  $\TAU$ measures the epoch's length in
      ``mutational'' time units, which equal $1/(2*u)$ generations.
      When time is measured this way, there is no need to specify a
      mutation rate.  A pair of individuals whose last common ancestor
      lived 3 units of mutational time ago will be separated (on
      average) by 3 mutations. 

\item[$K$]the number of subpopulations
\end{definition}
The structure of the file \io{pophist.ini} is very simple: Each row
corresponds to a different epoch, with the most recent epoch at the
top.  Each row contains four numbers, the values of $\THETA$, $M$,
$\TAU$, and $K$ (in that order).  The earliest epoch (in the bottom
row) is always infinite in length.  A \io{pophist.ini} file with three
epochs might look like this:
\begin{verbatim}
   %theta M   tau   K
    100   1   7     3
    1     0   0.001 1
    50    0.5 Inf   2
\end{verbatim}
The input routine interprets everything between ``\%'' and the end of
the line as a comment, a facility that I have used here to label the
columns.  The second line says that during the most recent
epoch, $\THETA=100$ (a large population), there were $M=1$ migrants
per generation between each pair of groups, the epoch lasted $\tau=7$
units of mutational time, and there were $K=3$ groups.

The third line refers to the epoch that ended 7 units of time ago, and
lasted only 0.001 units.  Thus, this epoch was very brief.  During
this epoch, the population was small $(\THETA=1)$ and contained only a
single group $(K=1)$.  The mutation rate is specified as $M=0$, but
this has no effect since there is only one group.

Line 4 refers to the initial epoch, during which the population was
large $(\THETA=50)$ and had two groups $(K=2)$, which exchanged
$M=0.5$ migrants per generation.  This epoch \emph{has} to be infinite
in length.  To make this clear, you should place ``Inf'' into the
bottom row of the \io{tau} column.  If you place some other
non-numeric string there, the program will abort with an error.  A
numeric value will generate a warning message and then treat the entry
as infinite anyway.

In summary, the example \io{pophist.ini} file describes a population
that was initially large, and divided into two groups connected by a
moderate level of migration.  The population underwent a bottleneck
during which the population was small and had only a single group.
Following the bottleneck, the population grew to twice its original
size and split into three groups.

Reductions in group numbers are accomplished by allowing randomly
chosen groups to go extinct rather than by merging them.

\section{Generating a genealogy}

The algorithm begins with the last epoch, which I denote as epoch $L$.
The $n$ individuals of the sample are at first divided evenly among
the $K[L]$ groups of epoch $L$.  Thus, the algorithm requires that $n$
be evenly divisible by $K[L]$.\footnote{The allocation of individuals
among groups in the simulation should match that in the data under
study.  Thus, the allocation used here is most appropriate when the
real data include samples of equal size, drawn from several groups.}

As the algorithm moves backward into the past, two types of event
occur.  Migrations occur when an individual moves from one group to
another, and ``coalescent events'' occur when two individuals have a
common ancestor and therefore coalesce to become a single individual.
The algorithm continues until only one individual is left.  The
mathematical details are summarized in
section~\ref{sec.coales.math}.

\section{Models of mutation\label{sec.mutation}}

The number of mutations along each branch is a Poisson random variable
with parameter $ut$, where $u$ is the mutation rate and $t$ the
length of the branch in generations \cite{Kimura:TPB-2-174}.  In
mutational time, branch lengths equal $\TAU \equiv 2ut$ and the
Poisson distribution has parameter $\TAU/2$.  Since the intervals of
the population history are specified in mutational time, there is no
need to specify a mutation rate.

The number of substitutions need not equal the number of mutations,
because mutations may strike the same site more than once.  To
determine the number of substitutions, the program must make some
assumption about the process of mutation.  The following models of
mutation are available.

\subsection{Model of infinite sites}

The model of infinite sites assumes that no site can mutate more than
once.  This assumption would follow if one had an infinite number of
sites, each with an infinetisimal mutation rate (hence the name).  No
one takes this to be an accurate description of any real mutational
process.  It is used because (1)~it makes the analysis easy and the
computer program fast, and because (2)~it often provides a good
approximation to more realistic models of mutation.  It does much
better at approximating the mismatch distribution and the things that
can be calculated from it (mean, variance, etc.) than at approximating
the number of segregating sites \cite{Rogers:MBE-13-895}.  It is the
default mutational model in the programs that do simulations.

\subsection{Model of finite sites with equal rates}

This model takes the number of sites to be finite but assumes that all
sites mutate at the same rate.  This model is invoked by the command
line argument \io{-mf}.

\subsection{Model of finite sites with Gamma-distributed rates}

The number of sites is finite, and the mutation rate at each site is
drawn independently from a gamma distribution.  The gamma distribution
has two parameters, a scale parameter $b$ and a shape parameter $c$.
The mutational time scale absorbs one of these $(b)$, but it is still
necessary to specify $c$.  This can be done on the command line by
specifying, say, \io{-g0.2}, to set $c=0.2$.  By default, $c=1$.

Some statistics, such as the number of segregating sites, are very
sensitive to the mutational model that is assumed.  This is bad, since
we are not able to specify the mutational process in any great detail.
Fortunately, the mismatch distribution is remarkably insensitive to
the mutational model.  With intraspecific human data, I get
essentially the same confidence region from all three models provided
that $c>0.01$.  Fortunately, the number of segregating sites in the
data allow us to reject the hypothesis that $c<0.05$, placing us
firmly in the region where the three mutational models all give the
same confidence regions.  For interspecific comparisons, mutation
models will become more important.

\section{Mathematical details of the coalescent
algorithm\label{sec.coales.math}} 

The hazard $h$ at time $\TAU$ is defined so that $h\, d\TAU$ is the
probability that an event of either type will occur between $\TAU$ and
$\TAU+d\TAU$, where $\TAU$ measures mutational time looking backwards
into the past.  The hazard depends on prevailing values of the
population history parameters, on the number of individuals, and on
how these are distributed among groups.  At any given time, let $s[j]$
denote the number of individuals within group $j$, $S = \sum_j s[j]$
(the total number of individuals), and $R\equiv \sum_j s[j]^2$ (the sum
of these numbers squared).  Then the hazard of an event
is%
%%%%%%%%%%%%%%%%%%%%%%%%%%%%%%%%%%%%%%%%%%%%%%%%%%%%%%%%%%%%%%%%
\footnote{Let $m$ denote the migration rate per generation, $g$
the group size, and $M \equiv m g$.  The hazard per
generation is
\begin{displaymath}
h^* \equiv\sum_j [s[j] m + s[j](s[j]-1)/(2g)]
= (1/g)[ S M + (R-S)/2]
\end{displaymath}
The cumulative hazard in $t$ generations is 
\begin{displaymath}
h^*t = \frac{2ut}{2ug}[S M + (R-S)/2] 
\equiv \frac{\TAU}{\gamma}[S M + (R-S)/2], 
\end{displaymath}
where $\TAU\equiv 2ut$ and $\gamma \equiv 2ug$.
Equation~\ref{eq.hazard} follows from the observation that, by
definition, the hazard $h$ in mutational time obeys $h\TAU \equiv h^*
t$.}
%%%%%%%%%%%%%%%%%%%%%%%%%%%%%%%%%%%%%%%%%%%%%%%%%%%%%%%%%%%%%%%%
\begin{equation}
\label{eq.hazard}
h = [S M[i] + (R - S)/2]/\gamma[i]
\end{equation}
where $\gamma[i] \equiv \theta[i]/K[i]$ and measures group size in epoch
$i$.

The algorithm first sets $S=n$, $R = K[L] (n/K[L])^2$, and then sets $h$
using these values together with the parameters of the final epoch,
$L$.  It then enters a loop that is executed repeatedly.  I describe
the steps of this loop briefly before describing each step in detail.

\paragraph{Overview of coalescent loop}
\begin{enumerate}
  \item \label{step.time} Find the time of the next event,
        changing epochs and recalculating $h$ as necessary. 
  \item \label{step.classify} Determine whether the next event is a
        migration or a coalescent event.
  \item \label{step.event} Carry out the next event.
\end{enumerate}
These steps are repeated until $S=1$.  Mutations are then added along
each branch.

\paragraph{Step~\ref{step.time}} Let $T[i]$ denote the amount of time
that we have already traveled (backwards) into epoch $i$.  To find the
time of the next event, draw a random number $x$ from an exponential
distribution whose parameter equals unity.  In a constant world, the
time of the next event would be $T[i] + x/h$.  If this time lies within
epoch $i$ (i.e.\ if $T[i] + x/h < \TAU[i]$), then we have found the time
of the next event.  Otherwise, change epochs as follows:
\begin{enumerate}
  \item[a] Subtract off the portion of $x$ that is ``used up'' by epoch
        $i$, i.e.\ subtract $h\cdot(\TAU[i] - T[i])$ from the value of
        $x$. 
  \item[b] Reset population history parameters to those of epoch $i-1$
        and set $T[i]$ to zero.  If $K[i-1] < K[i]$, join groups at
        random to diminish the number of groups.  If $K[i-1]>K[i]$,
        increase the number of groups, but allocate no individuals to the
        new groups.  Individuals will enter the new groups only through
        migration.%
%%%%%%%%%%%%%%%%%%%%%%%%%%%%%%%%%%%%%%%%%%%%%%%%%%%%%%%%%%%%%%%%
\footnote{The assumption for $K[i-1]>K[i]$ implies that, in forward
time, the number of groups has decreased because some groups have died
out.  Other assumptions are possible and the present one was chosen
only for computational convenience.
}
%%%%%%%%%%%%%%%%%%%%%%%%%%%%%%%%%%%%%%%%%%%%%%%%%%%%%%%%%%%%%%%%
  \item[c] Reset $R$ and $h$.  Subtract 1 from the value of $i$.
\end{enumerate}
This process repeats until $T[i] + x/h < \TAU[i]$.

\paragraph{Step~\ref{step.classify}}
Once the time of the next event has been established,
step~\ref{step.classify} classifies the event as either a migration or
a coalescent event.  Equation~\ref{eq.hazard} implies that the event is a
migration with probability
\begin{displaymath}
P = \frac{S M[i]}{S M[i] + (R - S)/2}
\end{displaymath}
Thus, step~\ref{step.classify} calls the next event a migration with
probability $P$ and a coalescent event with probability $1-P$.

\paragraph{Step~\ref{step.event}} If the next event is a migration,
then move a random individual into a new, randomly chosen group.  Then
reset $R$ and $h$.  

Otherwise, we have a coalescent event and the procedure is as follows.
First choose a group at random, weighting each group by the number of
pairs of individuals within it.  Then choose a random pair of
individuals from within the chosen group, replace the two individuals
with a single individual (their common ancestor), reduce $S$ by 1, and
reset $R$ and $h$.
