% -*-latex-*-
\chapter{Program: \io{normdist}\label{ch.normdist}}

\section{Purpose} Normalize (or un-normalize) mismatch distributions

\section{Usage\label{sec.normdist.usage}} \io{normdist [options] inputfile}\\
where options may include:
\begin{description}
\item[\io{-r$<$sampsize$>$}] Invert the usual operation? Default=NO
\item[\io{-2}]    Two-column output? Default=NO
\end{description}

\section{Description}

\section{Normalizing Mismatch Distributions}

The programs in this package demand that the mismatch distribution be
a list of integers, telling the number of pairs of individuals who
differ by $i$ sites, where $i=0,1,\ldots{}$.  In graphing such
distributions, on the other hand, it is usually better to normalize
the distribution so that it sums to unity.  The \io{normdist} program
is provided to simplify the tasks of (a)~normalizing distributions,
and (b)~converting normalizing distributions back to integer format.

To normalize a distribution, type
\begin{verbatim}
  normdist filename
\end{verbatim}
where \io{filename} is the name of a file containing a mismatch
distribution as a simple list of integers.  The program will type out
a normalized distribution.  To get the distribution in two-column
format, type
\begin{verbatim}
  normdist -2 filename
\end{verbatim}
If your distribution is already normalized, you may wish to convert it
back into integer format.  To do this, you must know the sample size.
Given this information, the reverse transformation is done by typing
\begin{verbatim}
  normdist -r<N> filename
\end{verbatim}

where \verb@<N>@ represents the sample size.

