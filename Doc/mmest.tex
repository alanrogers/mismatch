% -*-latex-*-
\chapter{Program: \io{mmest}\label{ch.mmest}}

\section{Purpose} Estimates parameters from mismatch data.

\section{Usage} \io{mmest [options] inputfile}\\
where options may include:
\begin{description}
\item[\io{ -l<x>}] Set length of mismatch distribution to
\verb@<x>@. Default: as in input.
\end{description}

\section{Input} Data are read from a file, which must be specified
on the command line.  The input file should look like:
\begin{verbatim}
% Source: Cann, RL, Stoneking, M, & Wilson, A. 1987. "Mitochondrial DNA
% and human evolution". Nature 325(1): 31-36.
segregating = 195;   %Polymorphic sites
nsites = 3000 ;      %Number of sites assayed
sampsize=147;        %Number of individuals in sample
histogram = 28 49 126 245 402 640 874 1178 1245 1119 1084 874 748 619
451 336 217 150 87 42 42 42 21 21 21 21 14 14 7 7 7 ;
\end{verbatim}
Comments begin with the \% character.  Assignment statements are of
form
\begin{verbatim}
  <label> = <values> ;
\end{verbatim}
where \verb@<label>@ is a name and \verb@<values>@ a blank-delimited
list of one or more values.  The input file must specify all of the
quantities shown in the example above.  \io{histogram} is the mismatch
distribution in integer format.  In other words, it is a vector whose
first entry gives the number of pairs differing by 0 sites, whose
second entry gives the number differing by 1 site, and so forth.  By
convention, \io{mmest}'s input files have suffix \io{mm}.  For
example, the input file shown above is called \io{csw.mm} in the
Mismatch distribution.

\section{Output} is self explanatory and includes estimates of:
\begin{description}
\item[$\THETAa$]the 2-parameter method of moments estimator described by
          Rogers \cite{Rogers:E-49-608}.
\item[$\THETAb$]estimated (very roughly) as $(1/F[0]) - 1$, where
      $F[0]$ is the first entry of the empirical mismatch distribution
      \cite{Rogers:MBE-9-552}. DO NOT PLACE ANY FAITH IN THIS
          ESTIMATOR; it provides only a very rough indication of the 
          magnitude of $\theta1$.  
\item[$\TAU$]the 2-parameter method of moments estimator described by
          Rogers \cite{Rogers:E-49-608}.
\item[$\MAE$] Mean Absolute Error.  The mean absolute difference
      between the fitted curve and the empirical distribution.  The
          empirical distribution is $h/\Sum{h}$ where $h$ is the histogram
          in the input data.  The fitted curve is calculated from
          equation~4 of Rogers and Harpending \cite{Rogers:MBE-9-552}. 
\item[Rghns]Harpending's \cite{Harpending:HB-66-591} $r$ statistic, a
measure of roughness.
\item[Seg]The number of segregating sites, copied from the input.
\end{description}
